\documentclass[a4paper, 11pt]{article}

\usepackage[utf8]{inputenc}
\usepackage[T1]{fontenc}
\usepackage{graphicx, wrapfig}
\usepackage[top=3cm, bottom=3cm, left=3.2cm, right=3.2cm]{geometry}
\usepackage{lmodern}
\usepackage{fancyhdr}
\usepackage{color, colortbl}
\usepackage[usenames, dvipsnames]{xcolor}
\usepackage{amsmath}
\usepackage{amssymb}
\usepackage{mathrsfs}
\usepackage{amsthm}
\usepackage{pgf, pgfplots, tikz}
\usepackage{listings}
\usepackage{makeidx}
\usepackage{hyperref}
\usepackage{setspace}
\usetikzlibrary{calc}
\usetikzlibrary{plotmarks}
\usetikzlibrary{decorations.markings}
\usetikzlibrary{decorations.pathreplacing}
\usepackage{manfnt, multicol}
\usepackage[section]{algorithm}
\usepackage{algorithmicx, algpseudocode, listings}
% fixes a bug for closing parantheses
\usepackage{etoolbox}
\makeatletter
\patchcmd{\lsthk@SelectCharTable}{%
  \lst@ifbreaklines\lst@Def{`)}{\lst@breakProcessOther)}\fi}{}{}{}
\makeatother
\usepackage{array, multirow, longtable}
\usepackage{enumerate}
\usepackage[inline]{enumitem}
\usepackage[english]{babel}
\usepackage{caption, tabu}


%°°°°°°°°°°°°°°°°°°°°°°°°°°°°°°°°°°°°°°°°°°°°°°%
%----Définition de l'environnement listings----%
%°°°°°°°°°°°°°°°°°°°°°°°°°°°°°°°°°°°°°°°°°°°°°°%

\lstset{literate=
  {á}{{\'a}}1 {é}{{\'e}}1 {í}{{\'i}}1 {ó}{{\'o}}1 {ú}{{\'u}}1
  {Á}{{\'A}}1 {É}{{\'E}}1 {Í}{{\'I}}1 {Ó}{{\'O}}1 {Ú}{{\'U}}1
  {à}{{\`a}}1 {è}{{\`e}}1 {ì}{{\`i}}1 {ò}{{\`o}}1 {ù}{{\`u}}1
  {À}{{\`A}}1 {È}{{\'E}}1 {Ì}{{\`I}}1 {Ò}{{\`O}}1 {Ù}{{\`U}}1
  {ä}{{\"a}}1 {ë}{{\"e}}1 {ï}{{\"i}}1 {ö}{{\"o}}1 {ü}{{\"u}}1
  {Ä}{{\"A}}1 {Ë}{{\"E}}1 {Ï}{{\"I}}1 {Ö}{{\"O}}1 {Ü}{{\"U}}1
  {â}{{\^a}}1 {ê}{{\^e}}1 {î}{{\^i}}1 {ô}{{\^o}}1 {û}{{\^u}}1
  {Â}{{\^A}}1 {Ê}{{\^E}}1 {Î}{{\^I}}1 {Ô}{{\^O}}1 {Û}{{\^U}}1
  {œ}{{\oe}}1 {Œ}{{\OE}}1 {æ}{{\ae}}1 {Æ}{{\AE}}1 {ß}{{\ss}}1
  {ç}{{\c c}}1 {Ç}{{\c C}}1 {ø}{{\o}}1 {å}{{\r a}}1 {Å}{{\r A}}1
  {€}{{\EUR}}1 {£}{{\pounds}}1
}

\lstdefinestyle{luaCode}{
  basicstyle=\footnotesize,        	% the size of the fonts that are used for the code
  breakatwhitespace=true,         	% sets if automatic breaks should only happen at whitespace
  breaklines=true,                 	% sets automatic line breaking
  captionpos=t,                    	% sets the caption-position to bottom
  commentstyle=\footnotesize\ttfamily\color{Emerald},    	% comment style
  deletekeywords={...},            	% if you want to delete keywords from the given language
  escapeinside={¦}{¦)},         	% if you want to add LaTeX within your code
  extendedchars=true,              	% lets you use non-ASCII characters; for 8-bits encodings only, does not work with UTF-8
  rulecolor=\color{OliveGreen!80},
  frame=trbl,
  framexleftmargin=10.8pt,
  framexrightmargin=7.2pt,
  keepspaces=true,                 	% keeps spaces in text, useful for keeping indentation of code (possibly needs columns=flexible)
  keywordstyle=\small\bfseries\color{OliveGreen!80},   % keyword style
  language=[5.0]Lua,                 	% the language of the code
  morekeywords={job.position},            	% if you want to add more keywords to the set
  numbers=left,                    	% where to put the line-numbers; possible values are (none, left, right)
  numbersep=3pt,                   	% how far the line-numbers are from the code
  numberstyle=\tiny\color{Gray}, 	% the style that is used for the line-numbers
  resetmargins=false                    % sets margins to 0 in a list
  showspaces=false,                	% show spaces everywhere adding particular underscores; it overrides 'showstringspaces'
  showstringspaces=false,          	% underline spaces within strings only
  showtabs=false,                  	% show tabs within strings adding particular underscores
  stepnumber=2,                    	% the step between two line-numbers. If it's 1, each line will be numbered
  stringstyle=\color{RawSienna},     	% string literal style
  tabsize=2,                       	% sets default tabsize to 2 spaces
  title=\lstname,                   	% show the filename of files included with \lstinputlisting; also try caption instead of title
  xleftmargin=1cm,                      % sets left margin indentation
  xrightmargin=1cm                      % sets right margin indentation
}

\lstdefinestyle{gpCode}{
  basicstyle=\footnotesize,        	% the size of the fonts that are used for the code
  breakatwhitespace=true,         	% sets if automatic breaks should only happen at whitespace
  breaklines=true,                 	% sets automatic line breaking
  captionpos=t,                    	% sets the caption-position to bottom
  commentstyle=\footnotesize\ttfamily\color{Emerald},    	% comment style
  deletekeywords={...},            	% if you want to delete keywords from the given language
  escapeinside={¦}{¦)},         	% if you want to add LaTeX within your code
  extendedchars=true,              	% lets you use non-ASCII characters; for 8-bits encodings only, does not work with UTF-8
  rulecolor=\color{OliveGreen!80},
  frame=trbl,
  framexleftmargin=10.8pt,
  framexrightmargin=7.2pt,
  keepspaces=true,                 	% keeps spaces in text, useful for keeping indentation of code (possibly needs columns=flexible)
  keywordstyle=\small\bfseries\color{OliveGreen!80},   % keyword style
  language=Gnuplot,                 	% the language of the code
  morekeywords={linestyle},            	% if you want to add more keywords to the set
  numbers=left,                    	% where to put the line-numbers; possible values are (none, left, right)
  numbersep=3pt,                   	% how far the line-numbers are from the code
  numberstyle=\tiny\color{Gray}, 	% the style that is used for the line-numbers
  resetmargins=false                    % sets margins to 0 in a list
  showspaces=false,                	% show spaces everywhere adding particular underscores; it overrides 'showstringspaces'
  showstringspaces=false,          	% underline spaces within strings only
  showtabs=false,                  	% show tabs within strings adding particular underscores
  stepnumber=2,                    	% the step between two line-numbers. If it's 1, each line will be numbered
  stringstyle=\color{RawSienna},     	% string literal style
  tabsize=2,                       	% sets default tabsize to 2 spaces
  title=\lstname,                   	% show the filename of files included with \lstinputlisting; also try caption instead of title
  xleftmargin=1cm,                      % sets left margin indentation
  xrightmargin=1cm                      % sets right margin indentation
}


\newcommand{\N}{\mathbb{N}}
\newcommand{\Q}{\mathbb{Q}} 
\newcommand{\R}{\mathbb{R}}
\newcommand{\Z}{\mathbb{Z}} 
\newcommand{\C}{\mathbb{C}}
\renewcommand{\epsilon}{\varepsilon} 
\renewcommand{\phi}{\varphi}


\newcommand{\CodeBrackets}[1]{{\color{RoyalBlue}{#1}}}    % color of parantheses and brackets
\newcommand{\CodeDigits}[1]{{\color{RawSienna}{#1}}} % color of numbers

%change les légendes des listings
\DeclareCaptionFont{white}{\color{white}}
\DeclareCaptionFormat{listing}{%
  \centering
  {\colorbox{OliveGreen!80}{%
      \parbox{0.9\linewidth}{%
          \centering
          #1#2#3
        }
      }
    }
  }
\captionsetup[lstlisting]{%
  format=listing,
  labelfont=white,
  textfont=white, 
  singlelinecheck=false, 
  indention=0pt,
  margin=0pt, 
  font={bf,sf,small}
}


%%%%%%%%	Définitions des environnements de théorèmes	%%%%%%%%
%% style théorème, lemme, proposition
\theoremstyle{plain}


%% style définitions, exemples et remarques
\theoremstyle{definition}


% For syncronisation with skim
\synctex = 1



%\title{{\bfseries Large-Scale Distributed Systems}\\ {\Large Project 1: Gossip-based dissemination, Peer
%    Sampling System}}
\title{%
  \normalfont{\bfseries{\rule{\linewidth}{2pt} Large-Scale Distributed Systems\\Project 3: Firefly-inspired synchronization\\ %
    \vspace{-0.4cm}  \rule{\linewidth}{2pt}}}
  }
\author{Laurent \textsc{Hayez}}
% Remove command to get current date 
\date{\today}


\begin{document}



\renewcommand{\proofname}{{\scshape Proof}}
\renewcommand{\labelitemi}{\textbullet}


\maketitle

\renewcommand{\contentsname}{Table of contents}
%Table of contents
\tableofcontents



%%  Section 1: Introduction
\section{Introduction}
\label{sec:introduction}

  In nature, fireflies produce light in order to attract mates or prey. One interesting feature of these
  beetles is that when they emit light in group, at some point, they do it in a synchronized manner, just by
  looking at when their neighbours emit light. This feature is interesting in large-scale distributed systems,
  as synchronization might be required, but one node does not know every other nodes. 

  The objective is thus to inspire ourselves from fireflies to try to synchronize nodes in a decentralized
  manner. At first we will detail the protocol skeleton and explain how the core of the protocols will
  work. Then we will look at a two models called ``phase-advance'' and ``phase-delay'' and briefly analyze
  them. The main and final part will be the ``adaptive Ermentrout model'' which is more representative of
  the reality. We will explain the implementation specifities and analyze this model in different situations.
  
  
%% Section 2: Skeleton

\section{Protocol skeleton}
\label{sec:impl-skel}

  According to the paper ``Firefly-inspired Heartbeat Synchronization in Overlay Networks'', the skeleton for
  the different algorithms is composed of two main functions, namely \textsc{activeThread} and
  \textsc{passiveThread}. We provide the pseudo code for the implementation in Algorithm
  \ref{alg:firefly-skeleton}.

  In the different protocols, a node is an oscillator characterized by its phase $\phi$ and the cycle length
  $\Delta$. We define $\phi$ as a sawtooth function with domain $[0,1]$ such that $\frac{d\phi}{dt} =
  \frac{1}{\Delta}$. This is represented in Figure \ref{fig:repr-phi}.

  When $\phi$ reaches $1$, the node will send a flash to a set of neighbour nodes, and $\phi$ is reset to
  $0$. The cycle length, depending on the model chosen, can be the same or different for all nodes. The
  function \textsc{updatePhi} will differ in our implementations, but we will come back on this when needed.

  The core of the synchronization protocol is the function \textsc{processFlash}, i.e. what a node does when it
  receives a flash. This function is responsible of how $\phi$ is updated. Depending on the implementation,
  $\phi$ will be updated, or $\Delta$ will be updated and will affect $\phi$.
  

  \begin{figure}[h]
    \centering
    \begin{tikzpicture}
      % Axis
      \draw[->, >=latex] (-1, 0) -- (10, 0) node[right]{time};
      \draw[->, >=latex] (0, -1) -- (0, 4);
      % Function phi
      \draw (0,0) -- (2, 2) (2, 0) -- (4, 2) (4, 0) -- (6, 2) (6, 0) -- (8, 2) (8, 0);
      \draw[dashed] (2,2) -- (2,0) (4, 2) -- (4, 0) (6, 2) -- (6, 0) (8,2) -- (8,0);
      \draw (8, 1) node[right]{$\phi$};
      % draw the fires
      \draw[color = BrickRed, ->, >=latex] (2,2) to node[near end, above, rotate=90]{Fire!} (2, 3);
      \draw[color = BrickRed, ->, >=latex] (4,2) to node[near end, above, rotate=90]{Fire!} (4, 3);
      \draw[color = BrickRed, ->, >=latex] (6,2) to node[near end, above, rotate=90]{Fire!} (6, 3);
      \draw[color = BrickRed, ->, >=latex] (8,2) to node[near end, above, rotate=90]{Fire!} (8, 3);
      % draw delta
      \draw[decorate, decoration = {brace, amplitude=10pt, mirror}, yshift = -4pt] (0,0) -- (2, 0) node [midway,
      yshift = -0.8cm] {$\Delta$};
      % draw tick 1 on y axis
      \draw (-0.1, 2) -- (0.1, 2) node[near start, left]{$1$};
      \draw[dashed, color=gray!60] (0,2) -- (10, 2);
      % draw dphi/dt = 1/delta
      \draw (1.5,1.5) node[above left, rotate=45]{$\frac{d\phi}{dt} = \frac{1}{\Delta}$};
    \end{tikzpicture}
    \caption{Representation of $\phi$ and its relation with $\Delta$}
    \label{fig:repr-phi}
  \end{figure}

  
   \begin{algorithm}
     \caption{Skeleton for the Firefly algorithms}
     \label{alg:firefly-skeleton}
     \begin{algorithmic}
       \State \textbf{Variables:}
       \State $\phi$ \Comment phase
       \State $\Delta$ \Comment cycle length
       \State update\_phi\_init $\gets$ false 
       \State update\_phi\_period $= \begin{cases} \frac{\Delta}{10} & \text{if } \Delta < 1 \\
         \frac{1}{10\Delta} & \text{if } \Delta \geq 1 \end{cases}$
       \State
       \Function{sendFlash()}{}
         \State $P \gets$ view from PSS
         \State send flash to all peers in $P$
       \EndFunction
       \State
       \Function{processFlash()}{}
         \State depends on the implementation
       \EndFunction
       \State
       \Function{updatePhi()}{}
         \If{$\phi < 1$}
           \State $\phi \gets \phi + \frac{1}{\Delta} \cdot $update\_phi\_period
         \Else
           \State fire event ``Flash!''
           \State $\phi \gets 0$
         \EndIf
       \EndFunction
       \State
       \Function{activeThread()}{}
          \If{$\neg$ update\_phi\_init}
            \State update\_phi\_init $\gets$ true
            \State new periodic thread ``updatePhi'' with period update\_phi\_period
          \EndIf
          \State wait for the event ``Flash!''
          \State sendFlash()
       \EndFunction
       \State
       \Function{passiveThread()}{}
         \State receive flash
         \State processFlash()
       \EndFunction
     \end{algorithmic}
   \end{algorithm}



\section{Conclusion}
\label{sec:conclusion}

  



    
    
    












	
\end{document}



%%% Local Variables:
%%% mode: latex
%%% TeX-master: t 
%%% End: